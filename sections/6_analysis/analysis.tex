\documentclass[../../main.tex]{subfiles}
\begin{document}


% Results
% Analysis

% Man kanske vill ha något här eller annanstans om andra sätt man kan kombinera two-scale och RTS, typ ha H-regions endast i R1 och R2 osv. 

Vi måste skriva hur stort time step vi har någonstans

%%%%%%%%
\section{Specific speedup}
%%%%%%%%
\textit{Why is the combined technique faster than two-scale, but not necessarily faster than RTS?}
It is all about the region determination step in two-scale algorithm. As one chooses the region for the high resolution it can be chosen to be more of less effective. However, because the region determined is directly coupled to the complexity of the scene this is no trivial decision. One can choose, performance wise, very expensive regions for high resolution. It can be illustrated by the following scenario:
An arbitrary sized scene is simulated with the two-scale resolution algorithm. To correctly adhere to the algorithm a high resolution area has to be determined. In this particular example the high resolution area is determined as the entire resolution. As an result we now have two simulations running to illustrate the same fluid. This potentially becomes very expensive and thus takes longer to simulate than RTS. This is true for the two-scale resolution algorithm, but because of the nature of the algorithm it is also true for the combined algorithm. In contrast, for the RTS algorithm it is less likely to make unique assumptions per scene that affects performance.  


%%%%%%%%
\subsection{Implementation }
%%%%%%%%
During our implementation we used the visual C++ 12.0 compiler on Windows 8.1 OS and compiled towards 64-bit. Tests were performed on computers with 16 gigabyte of memory with a Intel Core i7-2700K (3.5 GHz, 8 cores) and were multithreaded using OpenMP. 

Processor Intel Core i7-2700K (3.5 GHz, 8 cores) (H470-0004)

%%%%%%%%
\subsection{Discussion}
%%%%%%%%

Two-scale paper uses resolution factor 4 which gives them a higher speedup probably, alos they use a smaller H-region. 

\end{document}
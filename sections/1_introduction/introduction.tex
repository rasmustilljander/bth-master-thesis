\documentclass[../../main.tex]{subfiles}
\begin{document}
\tracingall

%%%%%%%%
\section{Fluid simulation}
%%%%%%%%
In the field of computer graphics, realism has always been desired. When it comes to fluid simulation there are multiple approaches to achieve realistic liquids. For the most part, the time cost associated with a realistic simulations is high. Therefore, research has been devoted to reduce total computational time. Methods for doing so are usually divided into three subgroups: Eulerian, Lagrangian, and a combination of the two.

Eulerian, or grid-based, methods is one alternative for simulating fluids in the computer graphics industry, it has high coherence with the ground reality. In Eulerian methods physical quantities (pressure, density, velocity, etc.) of the fluid are defined on a grid and are then changed over time, but the grid remains fixed. This can be visualized as observing a river pass by while sitting at the riverbank; properties of constant points in the water change continuously, but the observed position remains the same. 

Lagrangian methods, as opposed to the Eulerian, move fluid mass around explicitly. The quantities are tied to a small part of the fluid which is tracked through the whole simulation. An analogy would be sitting in a boat and drifting down a river while watching the water around the boat. These methods are usually more computationally expensive for higher details because a parcel of fluid needs to be directly aware of its surroundings which requires expensive neighborhood calculations. On the other hand, they offer advantages on simulating small scale features like droplets and splashes, because the surface is not bound to a grid but is instead represented by particles on the surface. In addition, Lagrangian methods conserve mass implicitly and do not need a separate scheme for mass conservation, because each particle represents a unit of mass and all particles are always tracked. 

Combinations of both Eulerian and Lagrangian methods utilizes both approaches using, for instance, a grid in one part of the fluid, typically the bottom, and particles for the surface.

Whether a simulation method is Eulerian or Lagrangian, it is progressed with a time step, a small value which is classically chosen globally and do not vary throughout the simulation. The time step determines how far the simulation is progressed in each iteration. A larger time step forwards the flow of the fluid further, and thus decreases the required total computational time, since fewer animation frames are performed to reach the same simulated time. Although a large time step seems desirable, choosing it too large could, for instance, cause two particles in the fluid to be progressed in such a way that they occupy the same space. This can cause physically incorrect movement and displeasing visual results and is referred to as instability or unstable simulation \citep{ihmsen2014sph}. Visually displeasing results means that explosions or waves occur where there should be none, while in a good visual result the fluid behaves as it would in real life. 

It is, however, still desirable to have as large as possible value for the time step but still maintain a stable simulation. While the total computational time changes, the time for each iteration stays the same. This implies that a simulation that uses a large time step can be simulated further with the same computational cost compared to the same simulation using a lower time step. Such a decrease in computational time is desirable and is what many techniques for simulating fluid seek to achieve, which will be made clear in the next chapter.

\end{document}
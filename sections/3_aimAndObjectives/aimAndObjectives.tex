\documentclass[../../main.tex]{subfiles}
\begin{document}
\tracingall

The aim of this thesis is to explore the possibilities of combining two methods designed to improve on two different aspects of PCISPH. To this end, we have defined two research questions: 

\begin{displayquote}
{\large \textbf{Q1.}} 
Is it possible to construct a combined technique based on the regional time stepping, which introduces an elevating effect on the time step, together with the two-scale simulation, which lowers particle count?
\end{displayquote} 

In order to answer this question we implemented both the two-scale algorithm as well as the regional time stepping algorithm, followed by the implementation of the combined algorithm. 

\begin{displayquote}
{\large \textbf{Q2.}} Will a combined technique simulate large body of fluids faster compared to its two base techniques and PCISPH?
\end{displayquote}

This question is tightly coupled to the first one, and to answer it we defined a set of test scenarios. In these scenarios we changed particle number, simulation time, and the scene configuration. Tests were then performed using all four techniques with same, or equivalent, parameters. 

\end{document}